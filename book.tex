\chapter{DR Cloud测试方案建议书}
\label{drcloud测试方案建议书}

\section{硬件配置要求}
\label{硬件配置要求}

\begin{table}[htbp]
\begin{minipage}{\linewidth}
\setlength{\tymax}{0.5\linewidth}
\centering
\small
\begin{tabulary}{\textwidth}{@{}LLLL@{}} \toprule
硬件名称&配置&数量&注释\\
\midrule
生产端容灾恢复一体机&双路英特尔至强E5 2630 v3系列或以上CPU32GB DDR4或以上内存至少4块2TB SATA HDD千兆网口4个万兆网口2个远程管理控制口&1台&用于模拟本地生产环境的保护\\
异地容灾恢复一体机&双路英特尔至强E5 2630 v3系列或以上CPU32GB DDR4或以上内存至少4块2TB SATA HDD千兆网口4个万兆网口2个远程管理控制口&1台&用于模拟异地生产环境的保护\\

\bottomrule

\end{tabulary}
\end{minipage}
\end{table}

\section{网络配置要求}
\label{网络配置要求}

\begin{itemize}
\item 两台物理机必须分别放置于两个VLAN中

\item 两个VLAN通过三层地址能够互相访问

\end{itemize}

\section{测试编号说明}
\label{测试编号说明}

\begin{table}[htbp]
\begin{minipage}{\linewidth}
\setlength{\tymax}{0.5\linewidth}
\centering
\small
\begin{tabulary}{\textwidth}{@{}LL@{}} \toprule
编号&说明\\
\midrule
DRCT0xx&安装部署相关的测试用例\\
DRCT1xx&保护相关的测试用例\\
DRCT2xx&恢复相关的测试用例\\
DRCT3xx&恢复相关的测试用例\\
DRCT5xx&平台本身测试用例\\

\bottomrule

\end{tabulary}
\end{minipage}
\end{table}

\section{测试用例}
\label{测试用例}

\begin{table}[htbp]
\begin{minipage}{\linewidth}
\setlength{\tymax}{0.5\linewidth}
\centering
\small
\begin{tabulary}{\textwidth}{@{}LLLL@{}} \toprule
编号&测试用例名称&说明&结果\\
\midrule
DRCT001&DR Cloud的安装和配置&\multicolumn{2}{l}{用于测试DR Cloud是否能正常安装,安装过程中不需要过多人为的介入}\\
DRCT101&Linux物理机客户端的安装和注册&\multicolumn{2}{l}{验证Linux Agent能够正确注册被保护的Linux物理机}\\
DRCT102&Windows物理机客户端的安装和注册&\multicolumn{2}{l}{验证Windows Agent能够正确注册被保护的Windows物理机}\\
DRCT103&VMware虚拟机注册&\multicolumn{2}{l}{能够正常添加ESXi或vCenter,发现所有虚拟机列表并且能正常注册}\\
DRCT104&为物理机设定本地保护&\multicolumn{2}{l}{正确将物理机加入本地保护}\\
DRCT105&为虚拟机设定本地保护&\multicolumn{2}{l}{能够为虚拟机正确加入保护,系统能够对虚拟机实现保护}\\
DRCT106&设置远程保护&\multicolumn{2}{l}{能够将本地的虚拟机正确复制到远程的DR Cloud中,在这个过程中,用户可以对数据采用压缩和加密的策略}\\
DRCT107&修改快照生成间隔&\multicolumn{2}{l}{能够正确修改快照生成的间隔,修改后,系统按照指定的时间间隔进行数据的保护}\\
DRCT108&手动触发快照&\multicolumn{2}{l}{能够手动的对被保护主机执行快照,执行后,系统立即进行快照}\\
DRCT201&物理机快速恢复至KVM平台&\multicolumn{2}{l}{正确将物理机恢复至KVM平台,恢复时间不超过30秒}\\
DRCT202&物理机快速恢复至VMware平台&\multicolumn{2}{l}{正确将物理机快速恢复至VMware平台,恢复时间不超过30秒}\\
DRCT203&VMware虚拟机快速恢复至VMware平台&\multicolumn{2}{l}{对VMware虚拟机可以采用快速启动的方式快速进行恢复,恢复时间不超过30秒}\\
DRCT204&将远程复制的物理机快速恢复至KVM&\multicolumn{2}{l}{将远程复制的物理机快速恢复至KVM平台,恢复时间不超过30秒}\\
DRCT205&将远程复制的VMware虚拟机快速恢复至VMware&\multicolumn{2}{l}{将远程复制的虚拟机正确恢复至VMware中,恢复时间不超过30秒}\\
DRCT301&SQL Server数据库保护与恢复测试&\multicolumn{2}{l}{保护SQL Server数据库,并且数据库可以正常恢复}\\
DRCT302&Windows Oracle数据库保护与恢复测试&\multicolumn{2}{l}{保护Oracle数据库,并且数据库可以正常恢复}\\
DRCT303&Linux MySQL数据库保护与恢复测试&\multicolumn{2}{l}{保护MySQL数据库,并且数据库可以正常恢复}\\
DRCT304&Linux Oracle数据库保护与恢复测试&\multicolumn{2}{l}{保护SQL Server数据库,并且数据库可以正常恢复}\\
DRCT501&设置每页表格显示的数量&\multicolumn{2}{l}{正确设置每页表格显示的数量}\\
DRCT502&语言设置&\multicolumn{2}{l}{正确设置系统界面的语言}\\

\bottomrule

\end{tabulary}
\end{minipage}
\end{table}

\section{DRCT001: DR Cloud的安装和配置}
\label{drct001:drcloud的安装和配置}

\begin{table}[htbp]
\begin{minipage}{\linewidth}
\setlength{\tymax}{0.5\linewidth}
\centering
\small
\begin{tabulary}{\textwidth}{@{}LL@{}} \toprule
目标&DR Cloud的安装和配置\\
\midrule
前提条件&DR Cloud ISO镜像\\
测试步骤&1、部署和软件并记录时间2、通过浏览器访问管理控制台\\
测试预期&1、在标准x86硬件上部署DR Cloud,可以部署在物理机或是虚拟机中2、不需要安装操作系统或数据库3、可以通过WEB浏览器远程访问管理控制台\\
\multicolumn{2}{l}{测试结果}\\
\multicolumn{2}{l}{注释}\\

\bottomrule

\end{tabulary}
\end{minipage}
\end{table}

\subsection{步骤1:部署和软件配置}
\label{步骤1:部署和软件配置}

1) 将ISO挂载或制作光盘,然后开机,当出现开始界面后,选择DRProphet + DRCloud

\begin{figure}[htbp]
\centering
\includegraphics[keepaspectratio,width=\textwidth,height=0.75\textheight]{images/DRCT001-1.png}
\end{figure}

2) 选择目标盘,使用tab键切换到Apply按钮并回车,开始安装过程

\begin{figure}[htbp]
\centering
\includegraphics[keepaspectratio,width=\textwidth,height=0.75\textheight]{images/DRCT001-2.png}
\end{figure}

3) 检查安装进度

\begin{figure}[htbp]
\centering
\includegraphics[keepaspectratio,width=\textwidth,height=0.75\textheight]{images/DRCT001-3.png}
\end{figure}

4) 安装完成后,按照提示将光盘取出并点击OK重启服务器

\begin{figure}[htbp]
\centering
\includegraphics[keepaspectratio,width=\textwidth,height=0.75\textheight]{images/DRCT001-4.png}
\end{figure}

5) 配置网络

当出现屏幕提示时,配置网络。注意:如果有多块网卡被检测到,将自动使用eth1网卡。

\begin{figure}[htbp]
\centering
\includegraphics[keepaspectratio,width=\textwidth,height=0.75\textheight]{images/DRCT001-5.png}
\end{figure}

6) 完成配置

选择skip,跳过federator的安装。

\begin{figure}[htbp]
\centering
\includegraphics[keepaspectratio,width=\textwidth,height=0.75\textheight]{images/DRCT001-6.png}
\end{figure}

步骤2:通过WEB访问管理控制台

\subsection{步骤2: 打开浏览器(推荐使用Chrome或者Firefox)}
\label{步骤2:打开浏览器推荐使用chrome或者firefox}

\begin{itemize}
\item 访问DR Cloud管理面板地址:http:\slash \slash :8088

\item 默认账户名:admin

\item 默认密码:admin\_pass

\end{itemize}

\section{DRCT101: Linux物理机客户端的安装和注册}
\label{drct101:linux物理机客户端的安装和注册}

\begin{table}[htbp]
\begin{minipage}{\linewidth}
\setlength{\tymax}{0.5\linewidth}
\centering
\small
\begin{tabulary}{\textwidth}{@{}LL@{}} \toprule
目标&安装DR Cloud Linux客户端并且在DR Cloud服务端注册\\
\midrule
前提条件&1、DR Cloud Linux客户端2、CentOS 7\\
测试步骤&1、在应用服务器安装DR Cloud Linux客户端2、在DR Cloud服务端确认是否已经注册,并且显示在列表中\\
测试预期&1、安装Linux DR Cloud不影响客户端2、成功向DR Cloud注册\\
\multicolumn{2}{l}{测试结果}\\
\multicolumn{2}{l}{注释}\\

\bottomrule

\end{tabulary}
\end{minipage}
\end{table}

\subsection{步骤1: 在应用服务器安装DR Cloud Linux客户端}
\label{步骤1:在应用服务器安装drcloudlinux客户端}

\begin{itemize}
\item 下载安装包

\end{itemize}

\begin{verbatim}
wget http://<管理节点IP>/softwares/centos7/egis-agent-1.1-1.el7.x86_64.rpm
wget http://<管理节点IP>/softwares/centos7/iscsi-initiator-utils-6.2.0.873-33.el7_2.2.x86_64.rpm
wget http://<管理节点IP>/softwares/centos7/iscsi-initiator-utils-iscsiuio-6.2.0.873-33.el7_2.2.x86_64.rpm
wget http://<管理节点IP>/softwares/centos7/dattobd-$(uname -r).rpm
wget http://<管理节点IP>/softwares/centos7/partclone-0.2.89-10.x86_64.rpm
\end{verbatim}

\begin{itemize}
\item 安装

\end{itemize}

\begin{verbatim}
rpm -ivh egis-agent-1.1-1.el7.x86_64.rpm \
  dattobd-$(uname -r).rpm \
  iscsi-initiator-utils-6.2.0.873-33.el7_2.2.x86_64.rpm  \
  partclone-0.2.89-10.x86_64.rpm
\end{verbatim}

\begin{itemize}
\item 设置保护

\end{itemize}

修改配置文件: \slash etc\slash sysconfig\slash egis-agent

\begin{verbatim}
log_dir=/var/log/egis-agent
drcloud_url=管理网IP:8760
\end{verbatim}

重新启动 egis-agent 服务,并检查服务状态,保证服务为 running 状态

\begin{verbatim}
systemctl restart egis-agent.service
systemctl status egis-agent.service
\end{verbatim}

\subsection{步骤2: 在DR Cloud服务端确认是否已经注册}
\label{步骤2:在drcloud服务端确认是否已经注册}

\begin{figure}[htbp]
\centering
\includegraphics[keepaspectratio,width=\textwidth,height=0.75\textheight]{images/DRCT101-1.png}
\end{figure}

\section{DRCT102: Windows物理机客户端的安装和注册}
\label{drct102:windows物理机客户端的安装和注册}

\begin{table}[htbp]
\begin{minipage}{\linewidth}
\setlength{\tymax}{0.5\linewidth}
\centering
\small
\begin{tabulary}{\textwidth}{@{}LL@{}} \toprule
目标&安装DR Cloud Windows客户端并且在DR Cloud服务端注册\\
\midrule
前提条件&1、DR Cloud Windows客户端2、dotnet framework或更高3、开启iscsi initiator4、防火墙打开3260和5988端口\\
测试步骤&1、在应用服务器安装DR Cloud Windows客户端2、填写注册信息(包括IO限速)3、在DR Cloud服务端确认是否已经注册,并且显示在列表中4、确认Agent能够正确上报被保护的主要磁盘5、设置限速\\
测试预期&1、安装Windows DR Cloud不影响客户端2、成功向DR Cloud注册3、向DR Cloud正确上报客户端磁盘信息\\
\multicolumn{2}{l}{测试结果}\\
\multicolumn{2}{l}{注释}\\

\bottomrule

\end{tabulary}
\end{minipage}
\end{table}

\subsection{步骤1: 安装DR Cloud Windows代理端}
\label{步骤1:安装drcloudwindows代理端}

1) 防火墙的设定

【开始】-$>$【命令提示符】执行:netsh advfirewall firewall add rule name=``DRP Backup Port'' dir=in action=allow protocol=TCP localport=5988

在防火墙中进行验证,验证方法:【开始】-$>$【管理工具】-$>$【高级安全Windows防火墙】

\begin{figure}[htbp]
\centering
\includegraphics[keepaspectratio,width=\textwidth,height=0.75\textheight]{images/DRCT102-1.png}
\end{figure}

在打开的界面中左侧列表中点击【入站规则】,在右侧窗口中应当显示【DRP Backup Port】,证明防火墙规则添加成功。

\begin{figure}[htbp]
\centering
\includegraphics[keepaspectratio,width=\textwidth,height=0.75\textheight]{images/DRCT102-2.png}
\end{figure}

2) Microsoft .NET Framework 4安装

打开浏览器,从容灾还原一体机上下载Microsoft .NET Framework 4安装包并且安装,下载地址:http:\slash \slash $<$管理节点IP$>$\slash softwares\slash windows\slash dotnet4.5.exe。

3) 安装保护代理

\begin{itemize}
\item 打开浏览器,从容灾还原一体机下载Windows保护Agent,下载地址:http:\slash \slash $<$管理节点IP$>$\slash softwares\slash windows\slash drprophet-master-agent--3.x-xxxx.exe

\item 运行安装,此时安装程序将安装三个相关的程序:DR Prophet Master Agent, DR Prophet Job Agent, DR Prophet Agent,采用默认安装方式,全部选择【Next】,如果出现文件目录已经存在的提示,则选择【OK】。

\end{itemize}

\begin{figure}[htbp]
\centering
\includegraphics[keepaspectratio,width=\textwidth,height=0.75\textheight]{images/DRCT102-3.png}
\end{figure}

\subsection{步骤2: 填写注册信息(包括IO限速)}
\label{步骤2:填写注册信息包括io限速}

安装成功后,将显示设置窗口(如下图所示)

\begin{itemize}
\item 链接方式:iscsi

\item 选择要保护的磁盘,把他们拖到``包含磁盘''中

\end{itemize}

\begin{figure}[htbp]
\centering
\includegraphics[keepaspectratio,width=\textwidth,height=0.75\textheight]{images/DRCT102-4.png}
\end{figure}

\begin{itemize}
\item 点击【Network】标签,在【Use the specified management server】中填入容灾还原一体机管理地址。

\item 如果需要对传输进行限速,则勾选【Throttling】,填入限速速度,默认情况下,无需限制。

\item 最后在【How can DRProphet server connect to this machine】中选择容灾还原一体机连接被保护服务器的IP地址。

\item 如果服务器原有IP地址可以直接连接,则选择【Connect to my local IP】。

\item 如果服务器需要其他地址转发后才能连接,则选择【Connect via a NAT address】。

\end{itemize}

\begin{figure}[htbp]
\centering
\includegraphics[keepaspectratio,width=\textwidth,height=0.75\textheight]{images/DRCT102-5.png}
\end{figure}

\begin{itemize}
\item 填写完成后,选择【OK】,完成客户端的配置。

\end{itemize}

\subsection{步骤3: 确认注册成功,以及上报信息无误。}
\label{步骤3:确认注册成功,以及上报信息无误。}

\begin{figure}[htbp]
\centering
\includegraphics[keepaspectratio,width=\textwidth,height=0.75\textheight]{images/DRCT102-6.png}
\end{figure}

\section{DRCT103: VMware虚拟机注册}
\label{drct103:vmware虚拟机注册}

\begin{table}[htbp]
\begin{minipage}{\linewidth}
\setlength{\tymax}{0.5\linewidth}
\centering
\small
\begin{tabulary}{\textwidth}{@{}LL@{}} \toprule
目标&发现VMware虚拟机并且在DR Cloud注册\\
\midrule
前提条件&1、ESXi 5.1或更高版本2、存储API功能确认开启\\
测试步骤&1、添加ESXi或者vCenter地址2、检查所有保护的虚拟机3、确认注册成功\\
测试预期&不需要代理程序就可以注册ESXi上的虚拟机\\
\multicolumn{2}{l}{测试结果}\\
\multicolumn{2}{l}{注释}\\

\bottomrule

\end{tabulary}
\end{minipage}
\end{table}

\subsection{步骤1: 添加ESXi或者vCenter地址}
\label{步骤1:添加esxi或者vcenter地址}

\begin{itemize}
\item 点击【虚拟化服务器】-$>$【VMware】-$>$【连接】-$>$【创建】-$>$ 输入要连接的Vcenter IP地址\slash 端口\slash 账户\slash 密码等信息-$>$【创建】

\end{itemize}

\begin{figure}[htbp]
\centering
\includegraphics[keepaspectratio,width=\textwidth,height=0.75\textheight]{images/DRCT103-1.png}
\end{figure}

\begin{itemize}
\item 创建完成后,点击【ESXI服务器】来查看此Vcenter下的所有ESXI主机,勾选ESXI主机 点击【网络设定】-$>$【添加】, 输入要创建的网络名称\slash VLAN ID-$>$【保存】

\end{itemize}

\begin{figure}[htbp]
\centering
\includegraphics[keepaspectratio,width=\textwidth,height=0.75\textheight]{images/DRCT103-2.png}
\end{figure}

\subsection{步骤2: 检查所有保护的虚拟机}
\label{步骤2:检查所有保护的虚拟机}

\begin{itemize}
\item 点击【虚拟化服务器】-$>$【VMware】-$>$【连接】点击添加的Vcenter主机名称,在跳转页面中会此Vcenter主机的连接详情以及其下的所有VM虚拟机

\end{itemize}

\begin{figure}[htbp]
\centering
\includegraphics[keepaspectratio,width=\textwidth,height=0.75\textheight]{images/DRCT103-3.png}
\end{figure}

\begin{itemize}
\item 勾选要保护的VM点击【注册】-$>$【确定】,勾选刚注册的主机【加入保护】-$>$【确定】

\end{itemize}

\begin{figure}[htbp]
\centering
\includegraphics[keepaspectratio,width=\textwidth,height=0.75\textheight]{images/DRCT103-4.png}
\end{figure}

\begin{figure}[htbp]
\centering
\includegraphics[keepaspectratio,width=\textwidth,height=0.75\textheight]{images/DRCT103-5.png}
\end{figure}

\subsection{步骤3: 确认注册成功}
\label{步骤3:确认注册成功}

在【未保护主机】中能够成功看到刚才加入保护的虚拟机

\begin{figure}[htbp]
\centering
\includegraphics[keepaspectratio,width=\textwidth,height=0.75\textheight]{images/DRCT103-6.png}
\end{figure}

\section{DRCT104: 为物理机设定本地保护}
\label{drct104:为物理机设定本地保护}

\begin{table}[htbp]
\begin{minipage}{\linewidth}
\setlength{\tymax}{0.5\linewidth}
\centering
\small
\begin{tabulary}{\textwidth}{@{}LL@{}} \toprule
目标&物理机本地保护的设定\\
\midrule
前提条件&DRCT102必须完成\\
测试步骤&1、通过DR Cloud管理平台设置保护2、确认保护设置成功\\
测试预期&1、成功添加保护2、所有镜像盘以thin provision方式分配\\
\multicolumn{2}{l}{测试结果}\\
\multicolumn{2}{l}{注释}\\

\bottomrule

\end{tabulary}
\end{minipage}
\end{table}

\subsection{步骤1: 通过DR Cloud管理平台设置保护}
\label{步骤1:通过drcloud管理平台设置保护}

\begin{itemize}
\item 登录DRP管理页面, http:\slash \slash $<$DRP\_SERVER\_IP$>$:8088

\item 点击【主机保护】-$>$【未保护主机】,勾选未保护主机点击【加入保护】选项

\end{itemize}

\begin{figure}[htbp]
\centering
\includegraphics[keepaspectratio,width=\textwidth,height=0.75\textheight]{images/DRCT104-1.png}
\end{figure}

\begin{itemize}
\item 选择后端存储服务器,可自动分配也可手动指定,点击【确定】选项

\end{itemize}

\begin{figure}[htbp]
\centering
\includegraphics[keepaspectratio,width=\textwidth,height=0.75\textheight]{images/DRCT104-2.png}
\end{figure}

\subsection{步骤2: 确认保护设置成功}
\label{步骤2:确认保护设置成功}

确认添加的主机已经成功的出现在【已保护主机中】

\begin{figure}[htbp]
\centering
\includegraphics[keepaspectratio,width=\textwidth,height=0.75\textheight]{images/DRCT104-3.png}
\end{figure}

\section{DRCT105:为虚拟机设定本地保护}
\label{drct105:为虚拟机设定本地保护}

\begin{itemize}
\item 登录DRP管理页面, http:\slash \slash $<$DRP\_SERVER\_IP$>$:8088

\item 点击【主机保护】-$>$【未保护主机】,勾选未保护主机点击【加入保护】选项

\end{itemize}

\begin{figure}[htbp]
\centering
\includegraphics[keepaspectratio,width=\textwidth,height=0.75\textheight]{images/DRCT104-1.png}
\end{figure}

\begin{itemize}
\item 选择后端存储服务器,可自动分配也可手动指定,点击【确定】选项

\end{itemize}

\begin{figure}[htbp]
\centering
\includegraphics[keepaspectratio,width=\textwidth,height=0.75\textheight]{images/DRCT104-2.png}
\end{figure}

\subsection{步骤2: 确认保护设置成功}
\label{步骤2:确认保护设置成功}

确认添加的主机已经成功的出现在【已保护主机中】

\begin{figure}[htbp]
\centering
\includegraphics[keepaspectratio,width=\textwidth,height=0.75\textheight]{images/DRCT104-3.png}
\end{figure}

\section{DRCT106:设置远程复制}
\label{drct106:设置远程复制}

\begin{table}[htbp]
\begin{minipage}{\linewidth}
\setlength{\tymax}{0.5\linewidth}
\centering
\small
\begin{tabulary}{\textwidth}{@{}LL@{}} \toprule
目标&为本地已经保护的节点设置远程复制\\
\midrule
前提条件&DRCT104和DRCT105必须完成\\
测试步骤&1、通过DR Cloud管理平台设置复制2、确认复制设置成功3、配置复制计划\\
测试预期&1、成功设定复制2、成功配置复制计划\\
\multicolumn{2}{l}{测试结果}\\
\multicolumn{2}{l}{注释}\\

\bottomrule

\end{tabulary}
\end{minipage}
\end{table}

\subsection{步骤1: 通过DR Cloud管理平台设置复制}
\label{步骤1:通过drcloud管理平台设置复制}

\begin{itemize}
\item 进入DR Cloud的【主机保护】-$>$【远程复制】,点击创建,【主机】选择已经保护的主机,【远程服务器】直接输入远端主机的IP信息,【存储用户】输入drprophet,【压缩】选择是,【加密】选择否,点击确认

\end{itemize}

\begin{figure}[htbp]
\centering
\includegraphics[keepaspectratio,width=\textwidth,height=0.75\textheight]{images/DRCT106-1.png}
\end{figure}

\subsection{步骤2: 确认复制设置成功}
\label{步骤2:确认复制设置成功}

\begin{itemize}
\item 在列表页面中应出现一条新的记录

\end{itemize}

\begin{figure}[htbp]
\centering
\includegraphics[keepaspectratio,width=\textwidth,height=0.75\textheight]{images/DRCT106-2.png}
\end{figure}

\subsection{步骤3: 配置复制计划}
\label{步骤3:配置复制计划}

\begin{itemize}
\item 选择刚刚保护的主机,点击更新,设置复制的时间间隔

\end{itemize}

\begin{figure}[htbp]
\centering
\includegraphics[keepaspectratio,width=\textwidth,height=0.75\textheight]{images/DRCT106-3.png}
\end{figure}

\section{DRCT107: 修改快照生成间隔}
\label{drct107:修改快照生成间隔}

\begin{table}[htbp]
\begin{minipage}{\linewidth}
\setlength{\tymax}{0.5\linewidth}
\centering
\small
\begin{tabulary}{\textwidth}{@{}LL@{}} \toprule
目标&修改快照生成间隔\\
\midrule
前提条件&DRCT104和DRCT105必须完成\\
测试步骤&1、修改快照计划2、确认数据复制开始时间正确\\
测试预期&1、成功修改快照计划2、正确配置快照开始时间\\
\multicolumn{2}{l}{测试结果}\\
\multicolumn{2}{l}{注释}\\

\bottomrule

\end{tabulary}
\end{minipage}
\end{table}

\subsection{步骤1: 修改快照计划}
\label{步骤1:修改快照计划}

\begin{itemize}
\item 选择【已保护主机】,勾选已经保护的主机,点击【保护设置】,设置【快照间隔】、【快照配额】和【快照开始时间】

\end{itemize}

\begin{figure}[htbp]
\centering
\includegraphics[keepaspectratio,width=\textwidth,height=0.75\textheight]{images/DRCT107-1.png}
\end{figure}

\subsection{步骤2: 确认数据复制开始时间正确}
\label{步骤2:确认数据复制开始时间正确}

\begin{itemize}
\item 在列表页面中,确认快照间隔设置正确

\end{itemize}

\begin{figure}[htbp]
\centering
\includegraphics[keepaspectratio,width=\textwidth,height=0.75\textheight]{images/DRCT107-1.png}
\end{figure}

\section{DRCT108: 手动触发快照}
\label{drct108:手动触发快照}

\begin{table}[htbp]
\begin{minipage}{\linewidth}
\setlength{\tymax}{0.5\linewidth}
\centering
\small
\begin{tabulary}{\textwidth}{@{}LL@{}} \toprule
目标&手动触发快照\\
\midrule
前提条件&DRCT104和DRCT105必须完成\\
测试步骤&1、手动触发快照2、查看作业列表,确认任务是否已经执行\\
测试预期&成功触发快照\\
\multicolumn{2}{l}{测试结果}\\
\multicolumn{2}{l}{注释}\\

\bottomrule

\end{tabulary}
\end{minipage}
\end{table}

\subsection{步骤1: 手动触发快照}
\label{步骤1:手动触发快照}

\begin{itemize}
\item 查看已保护主机,并勾选需要执行快照的主机,并点击创建快照

\end{itemize}

\begin{figure}[htbp]
\centering
\includegraphics[keepaspectratio,width=\textwidth,height=0.75\textheight]{images/DRCT108-1.png}
\end{figure}

\subsection{步骤2: 查看作业列表,确认任务是否已经执行}
\label{步骤2:查看作业列表,确认任务是否已经执行}

\begin{itemize}
\item 查看【运维管理】-$>$【任务】,查看任务是否已经开始执行

\end{itemize}

\begin{figure}[htbp]
\centering
\includegraphics[keepaspectratio,width=\textwidth,height=0.75\textheight]{images/DRCT108-2.png}
\end{figure}

\section{DRCT201: 物理机快速恢复至KVM平台}
\label{drct201:物理机快速恢复至kvm平台}

\begin{table}[htbp]
\begin{minipage}{\linewidth}
\setlength{\tymax}{0.5\linewidth}
\centering
\small
\begin{tabulary}{\textwidth}{@{}LL@{}} \toprule
目标&物理机快速恢复至KVM平台\\
\midrule
前提条件&1、DRCT104必须完成2、已经有快照产生\\
测试步骤&1、设定主机组2、设定恢复计划3、恢复4、查看已经恢复的资源\\
测试预期&1、成功设定主机组2、资源成功恢复3、恢复时间小于30秒\\
\multicolumn{2}{l}{测试结果}\\
\multicolumn{2}{l}{注释}\\

\bottomrule

\end{tabulary}
\end{minipage}
\end{table}

\subsection{步骤1: 设定主机组}
\label{步骤1:设定主机组}

\begin{itemize}
\item 进入DR Cloud,选择【灾难恢复】-$>$【主机组】,点击创建,选择主机组内包含的主机,选择已经保护的物理机,点击创建

\end{itemize}

\begin{figure}[htbp]
\centering
\includegraphics[keepaspectratio,width=\textwidth,height=0.75\textheight]{images/DRCT201-1.png}
\end{figure}

\begin{itemize}
\item 创建成功后,返回列表页面中应显示主机组的内容

\end{itemize}

\begin{figure}[htbp]
\centering
\includegraphics[keepaspectratio,width=\textwidth,height=0.75\textheight]{images/DRCT201-2.png}
\end{figure}

\subsection{步骤2: 设定恢复计划}
\label{步骤2:设定恢复计划}

\begin{itemize}
\item 选择【灾难恢复】-$>$【恢复计划】,点击创建,选择刚刚创建的主机组,系统将自动检测该主机组内包含的主机是否包含快照

\end{itemize}

\begin{figure}[htbp]
\centering
\includegraphics[keepaspectratio,width=\textwidth,height=0.75\textheight]{images/DRCT201-3.png}
\end{figure}

\begin{itemize}
\item 点击下一步,选择基本的配置信息,选择恢复的目标主机为KVM平台主机,并选择已经创建的网络和恢复的规格

\end{itemize}

\begin{figure}[htbp]
\centering
\includegraphics[keepaspectratio,width=\textwidth,height=0.75\textheight]{images/DRCT201-4.png}
\end{figure}

\begin{itemize}
\item 点击下一步,选择恢复的时间点信息,最后点击创建

\end{itemize}

\begin{figure}[htbp]
\centering
\includegraphics[keepaspectratio,width=\textwidth,height=0.75\textheight]{images/DRCT201-5.png}
\end{figure}

\subsection{步骤3: 恢复}
\label{步骤3:恢复}

\begin{itemize}
\item 在列表页面中,选择刚刚创建的恢复计划,并点击恢复,资源开始进行恢复

\end{itemize}

\begin{figure}[htbp]
\centering
\includegraphics[keepaspectratio,width=\textwidth,height=0.75\textheight]{images/DRCT201-6.png}
\end{figure}

\subsection{步骤4: 查看已经恢复的资源}
\label{步骤4:查看已经恢复的资源}

\begin{itemize}
\item 恢复完成后,点击【已恢复主机】

\end{itemize}

\begin{figure}[htbp]
\centering
\includegraphics[keepaspectratio,width=\textwidth,height=0.75\textheight]{images/DRCT201-7.png}
\end{figure}

\section{DRCT202:物理机快速恢复至VMware平台}
\label{drct202:物理机快速恢复至vmware平台}

\begin{table}[htbp]
\begin{minipage}{\linewidth}
\setlength{\tymax}{0.5\linewidth}
\centering
\small
\begin{tabulary}{\textwidth}{@{}LL@{}} \toprule
目标&物理机快速恢复至VMware平台\\
\midrule
前提条件&1、DRCT105必须完成2、已经有快照产生\\
测试步骤&1、设定主机组2、设定恢复计划3、恢复4、查看已经恢复的资源\\
测试预期&1、成功设定主机组2、资源成功恢复3、恢复时间小于30秒\\
\multicolumn{2}{l}{测试结果}\\
\multicolumn{2}{l}{注释}\\

\bottomrule

\end{tabulary}
\end{minipage}
\end{table}

\subsection{步骤1: 设定主机组}
\label{步骤1:设定主机组}

\begin{itemize}
\item 进入DR Cloud,选择【灾难恢复】-$>$【主机组】,点击创建,选择主机组内包含的主机,选择已经保护的Windows物理机,点击创建

\end{itemize}

\begin{quote}

注意:Linux物理机暂时无法恢复至VMware平台
\end{quote}

\begin{figure}[htbp]
\centering
\includegraphics[keepaspectratio,width=\textwidth,height=0.75\textheight]{images/DRCT201-1.png}
\end{figure}

\begin{itemize}
\item 创建成功后,返回列表页面中应显示主机组的内容

\end{itemize}

\begin{figure}[htbp]
\centering
\includegraphics[keepaspectratio,width=\textwidth,height=0.75\textheight]{images/DRCT201-2.png}
\end{figure}

\subsection{步骤2: 设定恢复计划}
\label{步骤2:设定恢复计划}

\begin{itemize}
\item 选择【灾难恢复】-$>$【恢复计划】,点击创建,选择刚刚创建的主机组,系统将自动检测该主机组内包含的主机是否包含快照

\end{itemize}

\begin{figure}[htbp]
\centering
\includegraphics[keepaspectratio,width=\textwidth,height=0.75\textheight]{images/DRCT201-3.png}
\end{figure}

\begin{itemize}
\item 点击下一步,选择基本的配置信息,选择恢复的目标主机为VMware的ESXi或者vCenter,并选择已经创建的网络和恢复的规格

\end{itemize}

\begin{figure}[htbp]
\centering
\includegraphics[keepaspectratio,width=\textwidth,height=0.75\textheight]{images/DRCT201-4.png}
\end{figure}

\begin{itemize}
\item 点击下一步,选择恢复的时间点信息,最后点击创建

\end{itemize}

\begin{figure}[htbp]
\centering
\includegraphics[keepaspectratio,width=\textwidth,height=0.75\textheight]{images/DRCT201-5.png}
\end{figure}

\subsection{步骤3: 恢复}
\label{步骤3:恢复}

\begin{itemize}
\item 在列表页面中,选择刚刚创建的恢复计划,并点击恢复,资源开始进行恢复

\end{itemize}

\begin{figure}[htbp]
\centering
\includegraphics[keepaspectratio,width=\textwidth,height=0.75\textheight]{images/DRCT201-6.png}
\end{figure}

\subsection{步骤4: 查看已经恢复的资源}
\label{步骤4:查看已经恢复的资源}

\begin{itemize}
\item 恢复完成后,点击【已恢复主机】

\end{itemize}

\begin{figure}[htbp]
\centering
\includegraphics[keepaspectratio,width=\textwidth,height=0.75\textheight]{images/DRCT201-7.png}
\end{figure}

\section{DRCT203: VMware虚拟机快速恢复至VMware平台}
\label{drct203:vmware虚拟机快速恢复至vmware平台}

\begin{table}[htbp]
\begin{minipage}{\linewidth}
\setlength{\tymax}{0.5\linewidth}
\centering
\small
\begin{tabulary}{\textwidth}{@{}LL@{}} \toprule
目标&VMware虚拟机快速恢复至VMware平台\\
\midrule
前提条件&1、DRCT105必须完成2、已经有快照产生\\
测试步骤&1、设定主机组2、设定恢复计划3、恢复4、查看已经恢复的资源\\
测试预期&1、成功设定主机组2、资源成功恢复3、恢复时间小于30秒\\
\multicolumn{2}{l}{测试结果}\\
\multicolumn{2}{l}{注释}\\

\bottomrule

\end{tabulary}
\end{minipage}
\end{table}

\subsection{步骤1: 设定主机组}
\label{步骤1:设定主机组}

\begin{itemize}
\item 进入DR Cloud,选择【灾难恢复】-$>$【主机组】,点击创建,选择主机组内包含的主机,选择已经保护的VMware虚拟机,点击创建

\end{itemize}

\begin{figure}[htbp]
\centering
\includegraphics[keepaspectratio,width=\textwidth,height=0.75\textheight]{images/DRCT203-1.png}
\end{figure}

\begin{itemize}
\item 创建成功后,返回列表页面中应显示主机组的内容

\end{itemize}

\begin{figure}[htbp]
\centering
\includegraphics[keepaspectratio,width=\textwidth,height=0.75\textheight]{images/DRCT201-2.png}
\end{figure}

\subsection{步骤2: 设定恢复计划}
\label{步骤2:设定恢复计划}

\begin{itemize}
\item 选择【灾难恢复】-$>$【恢复计划】,点击创建,选择刚刚创建的主机组,系统将自动检测该主机组内包含的主机是否包含快照

\end{itemize}

\begin{figure}[htbp]
\centering
\includegraphics[keepaspectratio,width=\textwidth,height=0.75\textheight]{images/DRCT201-3.png}
\end{figure}

\begin{itemize}
\item 点击下一步,选择基本的配置信息,选择恢复的目标主机为VMware的ESXi或者vCenter,并选择已经创建的网络和恢复的规格

\end{itemize}

\begin{figure}[htbp]
\centering
\includegraphics[keepaspectratio,width=\textwidth,height=0.75\textheight]{images/DRCT201-4.png}
\end{figure}

\begin{itemize}
\item 点击下一步,选择恢复的时间点信息,最后点击创建

\end{itemize}

\begin{figure}[htbp]
\centering
\includegraphics[keepaspectratio,width=\textwidth,height=0.75\textheight]{images/DRCT201-5.png}
\end{figure}

\subsection{步骤3: 恢复}
\label{步骤3:恢复}

\begin{itemize}
\item 在列表页面中,选择刚刚创建的恢复计划,并点击恢复,资源开始进行恢复

\end{itemize}

\begin{figure}[htbp]
\centering
\includegraphics[keepaspectratio,width=\textwidth,height=0.75\textheight]{images/DRCT201-6.png}
\end{figure}

\subsection{步骤4: 查看已经恢复的资源}
\label{步骤4:查看已经恢复的资源}

\begin{itemize}
\item 恢复完成后,点击【已恢复主机】

\end{itemize}

\begin{figure}[htbp]
\centering
\includegraphics[keepaspectratio,width=\textwidth,height=0.75\textheight]{images/DRCT201-7.png}
\end{figure}

\section{DRCT204: 将远程复制的物理机快速恢复至KVM}
\label{drct204:将远程复制的物理机快速恢复至kvm}

\begin{table}[htbp]
\begin{minipage}{\linewidth}
\setlength{\tymax}{0.5\linewidth}
\centering
\small
\begin{tabulary}{\textwidth}{@{}LL@{}} \toprule
目标&将远程复制的物理机快速恢复至KVM\\
\midrule
前提条件&1、DRCT104、DRCT105和DRCT106必须完成2、已经复制到远端DR Cloud\\
测试步骤&1、设定主机组2、设定恢复计划3、恢复4、查看已经恢复的资源\\
测试预期&1、成功设定主机组2、资源成功恢复3、恢复时间小于30秒\\
\multicolumn{2}{l}{测试结果}\\
\multicolumn{2}{l}{注释}\\

\bottomrule

\end{tabulary}
\end{minipage}
\end{table}

\subsection{步骤1: 设定主机组}
\label{步骤1:设定主机组}

\begin{itemize}
\item 进入DR Cloud,选择【灾难恢复】-$>$【主机组】,点击创建,选择主机组内包含的主机,选择已经保护的VMware虚拟机,点击创建

\end{itemize}

\begin{figure}[htbp]
\centering
\includegraphics[keepaspectratio,width=\textwidth,height=0.75\textheight]{images/DRCT204-1.png}
\end{figure}

\begin{itemize}
\item 创建成功后,返回列表页面中应显示主机组的内容

\end{itemize}

\begin{figure}[htbp]
\centering
\includegraphics[keepaspectratio,width=\textwidth,height=0.75\textheight]{images/DRCT201-2.png}
\end{figure}

\subsection{步骤2: 设定恢复计划}
\label{步骤2:设定恢复计划}

\begin{itemize}
\item 选择【灾难恢复】-$>$【恢复计划】,点击创建,选择刚刚创建的主机组,系统将自动检测该主机组内包含的主机是否包含快照

\end{itemize}

\begin{figure}[htbp]
\centering
\includegraphics[keepaspectratio,width=\textwidth,height=0.75\textheight]{images/DRCT201-3.png}
\end{figure}

\begin{itemize}
\item 点击下一步,选择基本的配置信息,选择恢复的目标主机为VMware的ESXi或者vCenter,并选择已经创建的网络和恢复的规格

\end{itemize}

\begin{figure}[htbp]
\centering
\includegraphics[keepaspectratio,width=\textwidth,height=0.75\textheight]{images/DRCT201-4.png}
\end{figure}

\begin{itemize}
\item 点击下一步,选择恢复的时间点信息,最后点击创建

\end{itemize}

\begin{figure}[htbp]
\centering
\includegraphics[keepaspectratio,width=\textwidth,height=0.75\textheight]{images/DRCT201-5.png}
\end{figure}

\subsection{步骤3: 恢复}
\label{步骤3:恢复}

\begin{itemize}
\item 在列表页面中,选择刚刚创建的恢复计划,并点击恢复,资源开始进行恢复

\end{itemize}

\begin{figure}[htbp]
\centering
\includegraphics[keepaspectratio,width=\textwidth,height=0.75\textheight]{images/DRCT201-6.png}
\end{figure}

\subsection{步骤4: 查看已经恢复的资源}
\label{步骤4:查看已经恢复的资源}

\begin{itemize}
\item 恢复完成后,点击【已恢复主机】

\end{itemize}

\begin{figure}[htbp]
\centering
\includegraphics[keepaspectratio,width=\textwidth,height=0.75\textheight]{images/DRCT201-7.png}
\end{figure}

\section{DRCT205: 将远程复制的VMware虚拟机快速恢复至VMware}
\label{drct205:将远程复制的vmware虚拟机快速恢复至vmware}

\begin{table}[htbp]
\begin{minipage}{\linewidth}
\setlength{\tymax}{0.5\linewidth}
\centering
\small
\begin{tabulary}{\textwidth}{@{}LL@{}} \toprule
目标&将远程复制的VMware虚拟机快速恢复至VMware\\
\midrule
前提条件&1、DRCT104、DRCT105和DRCT106必须完成2、已经复制到远端DR Cloud\\
测试步骤&1、设定主机组2、设定恢复计划3、恢复4、查看已经恢复的资源\\
测试预期&1、成功设定主机组2、资源成功恢复3、恢复时间小于30秒\\
\multicolumn{2}{l}{测试结果}\\
\multicolumn{2}{l}{注释}\\

\bottomrule

\end{tabulary}
\end{minipage}
\end{table}

\subsection{步骤1: 设定主机组}
\label{步骤1:设定主机组}

\begin{itemize}
\item 进入DR Cloud,选择【灾难恢复】-$>$【主机组】,点击创建,选择主机组内包含的主机,选择已经保护的VMware虚拟机,点击创建

\end{itemize}

\begin{figure}[htbp]
\centering
\includegraphics[keepaspectratio,width=\textwidth,height=0.75\textheight]{images/DRCT205-1.png}
\end{figure}

\begin{itemize}
\item 创建成功后,返回列表页面中应显示主机组的内容

\end{itemize}

\begin{figure}[htbp]
\centering
\includegraphics[keepaspectratio,width=\textwidth,height=0.75\textheight]{images/DRCT201-2.png}
\end{figure}

\subsection{步骤2: 设定恢复计划}
\label{步骤2:设定恢复计划}

\begin{itemize}
\item 选择【灾难恢复】-$>$【恢复计划】,点击创建,选择刚刚创建的主机组,系统将自动检测该主机组内包含的主机是否包含快照

\end{itemize}

\begin{figure}[htbp]
\centering
\includegraphics[keepaspectratio,width=\textwidth,height=0.75\textheight]{images/DRCT201-3.png}
\end{figure}

\begin{itemize}
\item 点击下一步,选择基本的配置信息,选择恢复的目标主机为VMware的ESXi或者vCenter,并选择已经创建的网络和恢复的规格

\end{itemize}

\begin{figure}[htbp]
\centering
\includegraphics[keepaspectratio,width=\textwidth,height=0.75\textheight]{images/DRCT201-4.png}
\end{figure}

\begin{itemize}
\item 点击下一步,选择恢复的时间点信息,最后点击创建

\end{itemize}

\begin{figure}[htbp]
\centering
\includegraphics[keepaspectratio,width=\textwidth,height=0.75\textheight]{images/DRCT201-5.png}
\end{figure}

\subsection{步骤3: 恢复}
\label{步骤3:恢复}

\begin{itemize}
\item 在列表页面中,选择刚刚创建的恢复计划,并点击恢复,资源开始进行恢复

\end{itemize}

\begin{figure}[htbp]
\centering
\includegraphics[keepaspectratio,width=\textwidth,height=0.75\textheight]{images/DRCT201-6.png}
\end{figure}

\subsection{步骤4: 查看已经恢复的资源}
\label{步骤4:查看已经恢复的资源}

\begin{itemize}
\item 恢复完成后,点击【已恢复主机】

\end{itemize}

\begin{figure}[htbp]
\centering
\includegraphics[keepaspectratio,width=\textwidth,height=0.75\textheight]{images/DRCT201-7.png}
\end{figure}

\section{DRCT301: SQL Server数据库保护与恢复测试}
\label{drct301:sqlserver数据库保护与恢复测试}

\begin{table}[htbp]
\begin{minipage}{\linewidth}
\setlength{\tymax}{0.5\linewidth}
\centering
\small
\begin{tabulary}{\textwidth}{@{}LL@{}} \toprule
目标&SQL Server数据库保护与恢复测试\\
\midrule
前提条件&1、在Windows 2008 R2 64bit服务器上,正确配置和安装SQL Server 2008版本2、可以通过客户端正确访问SQL Server3、在另外一台客户端上,已经安装了Benchmark Factory压力测试工具4、参考DRCT102完成Windows的注册和保护5、参考DRCT104将Windows加入保护\\
测试步骤&1、手动执行快照2、开始压力测试3、再次执行快照4、使用快照点进行恢复5、验证恢复的系统能否正常访问SQL Server\\
测试预期&1、SQL Server能够被正确保护2、SQL Server能被正常恢复\\
\multicolumn{2}{l}{测试结果}\\
\multicolumn{2}{l}{注释}\\

\bottomrule

\end{tabulary}
\end{minipage}
\end{table}

\subsection{步骤1: 手动执行快照}
\label{步骤1:手动执行快照}

\begin{itemize}
\item 查看已保护主机,并勾选需要执行快照的主机,并点击创建快照

\end{itemize}

\begin{figure}[htbp]
\centering
\includegraphics[keepaspectratio,width=\textwidth,height=0.75\textheight]{images/DRCT108-1.png}
\end{figure}

\subsection{步骤2: 开始压力测试}
\label{步骤2:开始压力测试}

\begin{itemize}
\item 在Benchmark Factory官方网站下载试用版本并正确安装:

\end{itemize}

\begin{quote}

https:\slash \slash www.quest.com\slash register\slash 54678\slash 
\end{quote}

\begin{itemize}
\item 配置连接方式

\item 配置压力测试的类型

\end{itemize}

\begin{quote}

注意:受被测试机器性能的限制,不建议选择过高的压力测试内容,避免服务器超载
\end{quote}

\begin{figure}[htbp]
\centering
\includegraphics[keepaspectratio,width=\textwidth,height=0.75\textheight]{images/DRCT301-1.png}
\end{figure}

\begin{itemize}
\item 确认压力测试开始

\end{itemize}

\begin{figure}[htbp]
\centering
\includegraphics[keepaspectratio,width=\textwidth,height=0.75\textheight]{images/DRCT301-2.png}
\end{figure}

\subsection{步骤3: 再次执行快照}
\label{步骤3:再次执行快照}

\begin{itemize}
\item 查看已保护主机,并勾选需要执行快照的主机,并点击创建快照

\end{itemize}

\begin{figure}[htbp]
\centering
\includegraphics[keepaspectratio,width=\textwidth,height=0.75\textheight]{images/DRCT108-1.png}
\end{figure}

\subsection{步骤4: 使用快照点进行恢复}
\label{步骤4:使用快照点进行恢复}

\begin{itemize}
\item 进入DR Cloud,选择【灾难恢复】-$>$【主机组】,点击创建,选择主机组内包含的主机,选择已经保护的物理机,点击创建

\end{itemize}

\begin{figure}[htbp]
\centering
\includegraphics[keepaspectratio,width=\textwidth,height=0.75\textheight]{images/DRCT201-1.png}
\end{figure}

\begin{itemize}
\item 创建成功后,返回列表页面中应显示主机组的内容

\end{itemize}

\begin{figure}[htbp]
\centering
\includegraphics[keepaspectratio,width=\textwidth,height=0.75\textheight]{images/DRCT201-2.png}
\end{figure}

\begin{itemize}
\item 选择【灾难恢复】-$>$【恢复计划】,点击创建,选择刚刚创建的主机组,系统将自动检测该主机组内包含的主机是否包含快照

\end{itemize}

\begin{figure}[htbp]
\centering
\includegraphics[keepaspectratio,width=\textwidth,height=0.75\textheight]{images/DRCT201-3.png}
\end{figure}

\begin{itemize}
\item 点击下一步,选择基本的配置信息,选择恢复的目标主机为KVM平台主机,并选择已经创建的网络和恢复的规格

\end{itemize}

\begin{figure}[htbp]
\centering
\includegraphics[keepaspectratio,width=\textwidth,height=0.75\textheight]{images/DRCT201-4.png}
\end{figure}

\begin{itemize}
\item 点击下一步,选择恢复的时间点信息,最后点击创建

\end{itemize}

\begin{figure}[htbp]
\centering
\includegraphics[keepaspectratio,width=\textwidth,height=0.75\textheight]{images/DRCT201-5.png}
\end{figure}

\begin{itemize}
\item 在列表页面中,选择刚刚创建的恢复计划,并点击恢复,资源开始进行恢复

\end{itemize}

\begin{figure}[htbp]
\centering
\includegraphics[keepaspectratio,width=\textwidth,height=0.75\textheight]{images/DRCT201-6.png}
\end{figure}

\begin{itemize}
\item 恢复完成后,点击【已恢复主机】

\end{itemize}

\begin{figure}[htbp]
\centering
\includegraphics[keepaspectratio,width=\textwidth,height=0.75\textheight]{images/DRCT201-7.png}
\end{figure}

\subsection{步骤5: 验证恢复的系统能够正常访问SQL Server}
\label{步骤5:验证恢复的系统能够正常访问sqlserver}

\section{DRCT302: Windows Oracle数据库保护与恢复测试}
\label{drct302:windowsoracle数据库保护与恢复测试}

\begin{table}[htbp]
\begin{minipage}{\linewidth}
\setlength{\tymax}{0.5\linewidth}
\centering
\small
\begin{tabulary}{\textwidth}{@{}LL@{}} \toprule
目标&Windows Oracle数据库保护与恢复测试\\
\midrule
前提条件&1、在Windows 2008 R2 64bit服务器上,正确配置和安装Oracle 12c版本2、可以通过客户端正确访问Oracle3、在另外一台客户端上,已经安装了Benchmark Factory压力测试工具4、参考DRCT102完成Windows的注册和保护5、参考DRCT104将Windows加入保护\\
测试步骤&1、手动执行快照2、开始压力测试3、再次执行快照4、使用快照点进行恢复5、验证恢复的系统能否正常访问Oracle\\
测试预期&1、Oracle能够被正确保护2、Oracle能被正常恢复\\
\multicolumn{2}{l}{测试结果}\\
\multicolumn{2}{l}{注释}\\

\bottomrule

\end{tabulary}
\end{minipage}
\end{table}

\subsection{步骤1: 手动执行快照}
\label{步骤1:手动执行快照}

\begin{itemize}
\item 查看已保护主机,并勾选需要执行快照的主机,并点击创建快照

\end{itemize}

\begin{figure}[htbp]
\centering
\includegraphics[keepaspectratio,width=\textwidth,height=0.75\textheight]{images/DRCT108-1.png}
\end{figure}

\subsection{步骤2: 开始压力测试}
\label{步骤2:开始压力测试}

\begin{itemize}
\item 在Benchmark Factory官方网站下载试用版本并正确安装:

\end{itemize}

\begin{quote}

https:\slash \slash www.quest.com\slash register\slash 54678\slash 
\end{quote}

\begin{itemize}
\item 下载Oracle的客户端

\end{itemize}

在客户端安装Oracle客户端,访问Oracle Instance Client Downloads for Microsoft Windows(x64)页面:

\begin{quote}

http:\slash \slash www.oracle.com\slash technetwork\slash topics\slash winx64soft--089540.html
\end{quote}

\begin{figure}[htbp]
\centering
\includegraphics[keepaspectratio,width=\textwidth,height=0.75\textheight]{images/DRCT302-1.png}
\end{figure}

Accept协议后,下载instanceclient-basic-windows.x64--12.1.0.2.0.zip。

\begin{itemize}
\item 配置Oracle客户端

\end{itemize}

下载完成后,将zip文件解压缩到C盘根目录中,例如将所有文件解压缩到C:\textbackslash{}instantclient\_12\_1中后,配置系统环境变量,在PATH中添加这个目录。

\begin{figure}[htbp]
\centering
\includegraphics[keepaspectratio,width=\textwidth,height=0.75\textheight]{images/DRCT302-2.png}
\end{figure}

\begin{itemize}
\item 使用Benchmark连接Oracle

\end{itemize}

\begin{figure}[htbp]
\centering
\includegraphics[keepaspectratio,width=\textwidth,height=0.75\textheight]{images/DRCT302-3.png}
\end{figure}

\begin{itemize}
\item 配置压力测试的类型

\end{itemize}

\begin{quote}

注意:受被测试机器性能的限制,不建议选择过高的压力测试内容,避免服务器超载
\end{quote}

\begin{figure}[htbp]
\centering
\includegraphics[keepaspectratio,width=\textwidth,height=0.75\textheight]{images/DRCT301-1.png}
\end{figure}

\begin{itemize}
\item 确认压力测试开始

\end{itemize}

\begin{figure}[htbp]
\centering
\includegraphics[keepaspectratio,width=\textwidth,height=0.75\textheight]{images/DRCT301-2.png}
\end{figure}

\subsection{步骤3: 再次执行快照}
\label{步骤3:再次执行快照}

\begin{itemize}
\item 查看已保护主机,并勾选需要执行快照的主机,并点击创建快照

\end{itemize}

\begin{figure}[htbp]
\centering
\includegraphics[keepaspectratio,width=\textwidth,height=0.75\textheight]{images/DRCT108-1.png}
\end{figure}

\subsection{步骤4: 使用快照点进行恢复}
\label{步骤4:使用快照点进行恢复}

\begin{itemize}
\item 进入DR Cloud,选择【灾难恢复】-$>$【主机组】,点击创建,选择主机组内包含的主机,选择已经保护的物理机,点击创建

\end{itemize}

\begin{figure}[htbp]
\centering
\includegraphics[keepaspectratio,width=\textwidth,height=0.75\textheight]{images/DRCT201-1.png}
\end{figure}

\begin{itemize}
\item 创建成功后,返回列表页面中应显示主机组的内容

\end{itemize}

\begin{figure}[htbp]
\centering
\includegraphics[keepaspectratio,width=\textwidth,height=0.75\textheight]{images/DRCT201-2.png}
\end{figure}

\begin{itemize}
\item 选择【灾难恢复】-$>$【恢复计划】,点击创建,选择刚刚创建的主机组,系统将自动检测该主机组内包含的主机是否包含快照

\end{itemize}

\begin{figure}[htbp]
\centering
\includegraphics[keepaspectratio,width=\textwidth,height=0.75\textheight]{images/DRCT201-3.png}
\end{figure}

\begin{itemize}
\item 点击下一步,选择基本的配置信息,选择恢复的目标主机为KVM平台主机,并选择已经创建的网络和恢复的规格

\end{itemize}

\begin{figure}[htbp]
\centering
\includegraphics[keepaspectratio,width=\textwidth,height=0.75\textheight]{images/DRCT201-4.png}
\end{figure}

\begin{itemize}
\item 点击下一步,选择恢复的时间点信息,最后点击创建

\end{itemize}

\begin{figure}[htbp]
\centering
\includegraphics[keepaspectratio,width=\textwidth,height=0.75\textheight]{images/DRCT201-5.png}
\end{figure}

\begin{itemize}
\item 在列表页面中,选择刚刚创建的恢复计划,并点击恢复,资源开始进行恢复

\end{itemize}

\begin{figure}[htbp]
\centering
\includegraphics[keepaspectratio,width=\textwidth,height=0.75\textheight]{images/DRCT201-6.png}
\end{figure}

\begin{itemize}
\item 恢复完成后,点击【已恢复主机】

\end{itemize}

\begin{figure}[htbp]
\centering
\includegraphics[keepaspectratio,width=\textwidth,height=0.75\textheight]{images/DRCT201-7.png}
\end{figure}

\subsection{步骤5: 验证恢复的系统能够正常访问Oracle}
\label{步骤5:验证恢复的系统能够正常访问oracle}

\section{DRCT303: Linux MySQL数据库保护与恢复测试}
\label{drct303:linuxmysql数据库保护与恢复测试}

\begin{table}[htbp]
\begin{minipage}{\linewidth}
\setlength{\tymax}{0.5\linewidth}
\centering
\small
\begin{tabulary}{\textwidth}{@{}LL@{}} \toprule
目标&Linux MySQL数据库保护与恢复测试\\
\midrule
前提条件&1、CentOS7 64bit服务器上,正确配置和安装MySQL2、可以通过客户端正确访问MySQL3、在另外一台CentOS 7客户端上,已经安装了sysbench压力测试工具4、参考DRCT101完成Linux的注册和保护5、参考DRCT104将Linux加入保护\\
测试步骤&1、手动执行快照2、开始压力测试3、再次执行快照4、使用快照点进行恢复5、验证恢复的系统能否正常访问MySQL\\
测试预期&1、MySQL能够被正确保护2、MySQL能被正常恢复\\
\multicolumn{2}{l}{测试结果}\\
\multicolumn{2}{l}{注释}\\

\bottomrule

\end{tabulary}
\end{minipage}
\end{table}

\subsection{步骤1: 手动执行快照}
\label{步骤1:手动执行快照}

\begin{itemize}
\item 查看已保护主机,并勾选需要执行快照的主机,并点击创建快照

\end{itemize}

\begin{figure}[htbp]
\centering
\includegraphics[keepaspectratio,width=\textwidth,height=0.75\textheight]{images/DRCT108-1.png}
\end{figure}

\subsection{步骤2: 开始压力测试}
\label{步骤2:开始压力测试}

\begin{itemize}
\item 在CentOS 7客户端安装sysbench

\end{itemize}

\begin{quote}

注意:使用Ubuntu 14.04的客户端可以直接使用sudo apt-get install -y sysbench
\end{quote}

\begin{verbatim}
$  wget http://nchc.dl.sourceforge.net/project/sysbench/sysb
ench/0.4.12/sysbench-0.4.12.tar.gz
$ tar zxf sysbench-0.4.12.tar.gz
$ cd sysbench-0.4.12
$ ./configure -with-mysql-includes=/usr/include/mysql \
--with-mysql-libs=/var/lib/mysql
$ make && make install
\end{verbatim}

\begin{itemize}
\item MySQL数据库需要开放远程访问的权限

\end{itemize}

\begin{verbatim}
$ grant all privileges on *.* to ‘root’@’%’ identified by ‘Your_password’;
$ flush privileges; 
\end{verbatim}

\begin{itemize}
\item 在客户端使用sysbench产生压力

\end{itemize}

\begin{verbatim}
$ sysbench -test=oltp -max-time=60 -thread-stack-size=10000 -mysql-user=<mysql_user> -mysql-password=<mysql_password> -mysql-host=<mysql_ip>  -mysql-port=3306 prepare
$ sysbench --test=oltp --max-time=60 --thread-stack-size=10000 --mysql-user=<mysql_user>  --mysql-password=<mysql_password>
 --mysql-host= <test_host_ip>  --mysql-port=3306 run
\end{verbatim}

\subsection{步骤3: 再次执行快照}
\label{步骤3:再次执行快照}

\begin{itemize}
\item 查看已保护主机,并勾选需要执行快照的主机,并点击创建快照

\end{itemize}

\begin{figure}[htbp]
\centering
\includegraphics[keepaspectratio,width=\textwidth,height=0.75\textheight]{images/DRCT108-1.png}
\end{figure}

\subsection{步骤4: 使用快照点进行恢复}
\label{步骤4:使用快照点进行恢复}

\begin{itemize}
\item 进入DR Cloud,选择【灾难恢复】-$>$【主机组】,点击创建,选择主机组内包含的主机,选择已经保护的物理机,点击创建

\end{itemize}

\begin{figure}[htbp]
\centering
\includegraphics[keepaspectratio,width=\textwidth,height=0.75\textheight]{images/DRCT201-1.png}
\end{figure}

\begin{itemize}
\item 创建成功后,返回列表页面中应显示主机组的内容

\end{itemize}

\begin{figure}[htbp]
\centering
\includegraphics[keepaspectratio,width=\textwidth,height=0.75\textheight]{images/DRCT201-2.png}
\end{figure}

\begin{itemize}
\item 选择【灾难恢复】-$>$【恢复计划】,点击创建,选择刚刚创建的主机组,系统将自动检测该主机组内包含的主机是否包含快照

\end{itemize}

\begin{figure}[htbp]
\centering
\includegraphics[keepaspectratio,width=\textwidth,height=0.75\textheight]{images/DRCT201-3.png}
\end{figure}

\begin{itemize}
\item 点击下一步,选择基本的配置信息,选择恢复的目标主机为KVM平台主机,并选择已经创建的网络和恢复的规格

\end{itemize}

\begin{figure}[htbp]
\centering
\includegraphics[keepaspectratio,width=\textwidth,height=0.75\textheight]{images/DRCT201-4.png}
\end{figure}

\begin{itemize}
\item 点击下一步,选择恢复的时间点信息,最后点击创建

\end{itemize}

\begin{figure}[htbp]
\centering
\includegraphics[keepaspectratio,width=\textwidth,height=0.75\textheight]{images/DRCT201-5.png}
\end{figure}

\begin{itemize}
\item 在列表页面中,选择刚刚创建的恢复计划,并点击恢复,资源开始进行恢复

\end{itemize}

\begin{figure}[htbp]
\centering
\includegraphics[keepaspectratio,width=\textwidth,height=0.75\textheight]{images/DRCT201-6.png}
\end{figure}

\subsection{步骤4: 查看已经恢复的资源}
\label{步骤4:查看已经恢复的资源}

\begin{itemize}
\item 恢复完成后,点击【已恢复主机】

\end{itemize}

\begin{figure}[htbp]
\centering
\includegraphics[keepaspectratio,width=\textwidth,height=0.75\textheight]{images/DRCT201-7.png}
\end{figure}

\subsection{步骤5: 验证恢复的系统能够正常访问MySQL}
\label{步骤5:验证恢复的系统能够正常访问mysql}

\section{DRCT304: Linux Oracle数据库保护与恢复测试}
\label{drct304:linuxoracle数据库保护与恢复测试}

\begin{table}[htbp]
\begin{minipage}{\linewidth}
\setlength{\tymax}{0.5\linewidth}
\centering
\small
\begin{tabulary}{\textwidth}{@{}LL@{}} \toprule
目标&Windows Oracle数据库保护与恢复测试\\
\midrule
前提条件&1、在CentOS 7 64bit服务器上,正确配置和安装Oracle 12c版本2、可以通过客户端正确访问Oracle3、在另外一台客户端上,已经安装了Benchmark Factory压力测试工具4、参考DRCT101完成Linux的注册和保护5、参考DRCT104将Linux加入保护\\
测试步骤&1、手动执行快照2、开始压力测试3、再次执行快照4、使用快照点进行恢复5、验证恢复的系统能否正常访问Oracle\\
测试预期&1、Oracle能够被正确保护2、Oracle能被正常恢复\\
\multicolumn{2}{l}{测试结果}\\
\multicolumn{2}{l}{注释}\\

\bottomrule

\end{tabulary}
\end{minipage}
\end{table}

\subsection{步骤1: 手动执行快照}
\label{步骤1:手动执行快照}

\begin{itemize}
\item 查看已保护主机,并勾选需要执行快照的主机,并点击创建快照

\end{itemize}

\begin{figure}[htbp]
\centering
\includegraphics[keepaspectratio,width=\textwidth,height=0.75\textheight]{images/DRCT108-1.png}
\end{figure}

\subsection{步骤2: 开始压力测试}
\label{步骤2:开始压力测试}

\begin{itemize}
\item 在Benchmark Factory官方网站下载试用版本并正确安装:

\end{itemize}

\begin{quote}

https:\slash \slash www.quest.com\slash register\slash 54678\slash 
\end{quote}

\begin{itemize}
\item 下载Oracle的客户端

\end{itemize}

在客户端安装Oracle客户端,访问Oracle Instance Client Downloads for Microsoft Windows(x64)页面:

\begin{quote}

http:\slash \slash www.oracle.com\slash technetwork\slash topics\slash winx64soft--089540.html
\end{quote}

\begin{figure}[htbp]
\centering
\includegraphics[keepaspectratio,width=\textwidth,height=0.75\textheight]{images/DRCT302-1.png}
\end{figure}

Accept协议后,下载instanceclient-basic-windows.x64--12.1.0.2.0.zip。

\begin{itemize}
\item 配置Oracle客户端

\end{itemize}

下载完成后,将zip文件解压缩到C盘根目录中,例如将所有文件解压缩到C:\textbackslash{}instantclient\_12\_1中后,配置系统环境变量,在PATH中添加这个目录。

\begin{figure}[htbp]
\centering
\includegraphics[keepaspectratio,width=\textwidth,height=0.75\textheight]{images/DRCT302-2.png}
\end{figure}

\begin{itemize}
\item 使用Benchmark连接Oracle

\end{itemize}

\begin{figure}[htbp]
\centering
\includegraphics[keepaspectratio,width=\textwidth,height=0.75\textheight]{images/DRCT302-3.png}
\end{figure}

\begin{itemize}
\item 配置压力测试的类型

\end{itemize}

\begin{quote}

注意:受被测试机器性能的限制,不建议选择过高的压力测试内容,避免服务器超载
\end{quote}

\begin{figure}[htbp]
\centering
\includegraphics[keepaspectratio,width=\textwidth,height=0.75\textheight]{images/DRCT301-1.png}
\end{figure}

\begin{itemize}
\item 确认压力测试开始

\end{itemize}

\begin{figure}[htbp]
\centering
\includegraphics[keepaspectratio,width=\textwidth,height=0.75\textheight]{images/DRCT301-2.png}
\end{figure}

\subsection{步骤3: 再次执行快照}
\label{步骤3:再次执行快照}

\begin{itemize}
\item 查看已保护主机,并勾选需要执行快照的主机,并点击创建快照

\end{itemize}

\begin{figure}[htbp]
\centering
\includegraphics[keepaspectratio,width=\textwidth,height=0.75\textheight]{images/DRCT108-1.png}
\end{figure}

\subsection{步骤4: 使用快照点进行恢复}
\label{步骤4:使用快照点进行恢复}

\begin{itemize}
\item 进入DR Cloud,选择【灾难恢复】-$>$【主机组】,点击创建,选择主机组内包含的主机,选择已经保护的物理机,点击创建

\end{itemize}

\begin{figure}[htbp]
\centering
\includegraphics[keepaspectratio,width=\textwidth,height=0.75\textheight]{images/DRCT201-1.png}
\end{figure}

\begin{itemize}
\item 创建成功后,返回列表页面中应显示主机组的内容

\end{itemize}

\begin{figure}[htbp]
\centering
\includegraphics[keepaspectratio,width=\textwidth,height=0.75\textheight]{images/DRCT201-2.png}
\end{figure}

\begin{itemize}
\item 选择【灾难恢复】-$>$【恢复计划】,点击创建,选择刚刚创建的主机组,系统将自动检测该主机组内包含的主机是否包含快照

\end{itemize}

\begin{figure}[htbp]
\centering
\includegraphics[keepaspectratio,width=\textwidth,height=0.75\textheight]{images/DRCT201-3.png}
\end{figure}

\begin{itemize}
\item 点击下一步,选择基本的配置信息,选择恢复的目标主机为KVM平台主机,并选择已经创建的网络和恢复的规格

\end{itemize}

\begin{figure}[htbp]
\centering
\includegraphics[keepaspectratio,width=\textwidth,height=0.75\textheight]{images/DRCT201-4.png}
\end{figure}

\begin{itemize}
\item 点击下一步,选择恢复的时间点信息,最后点击创建

\end{itemize}

\begin{figure}[htbp]
\centering
\includegraphics[keepaspectratio,width=\textwidth,height=0.75\textheight]{images/DRCT201-5.png}
\end{figure}

\begin{itemize}
\item 在列表页面中,选择刚刚创建的恢复计划,并点击恢复,资源开始进行恢复

\end{itemize}

\begin{figure}[htbp]
\centering
\includegraphics[keepaspectratio,width=\textwidth,height=0.75\textheight]{images/DRCT201-6.png}
\end{figure}

\begin{itemize}
\item 恢复完成后,点击【已恢复主机】

\end{itemize}

\begin{figure}[htbp]
\centering
\includegraphics[keepaspectratio,width=\textwidth,height=0.75\textheight]{images/DRCT201-7.png}
\end{figure}

\subsection{步骤5: 验证恢复的系统能够正常访问Oracle}
\label{步骤5:验证恢复的系统能够正常访问oracle}

\section{DRCT501: 设置每页表格显示的数量}
\label{drct501:设置每页表格显示的数量}

\begin{table}[htbp]
\begin{minipage}{\linewidth}
\setlength{\tymax}{0.5\linewidth}
\centering
\small
\begin{tabulary}{\textwidth}{@{}LL@{}} \toprule
目标&设置每页表格显示的数量\\
\midrule
\multicolumn{2}{l}{前提条件}\\
测试步骤&1、设置每页显示表格的数量2、查看列表页面中表格的显示数量是否改变\\
测试预期&成功设置表格每页显示的数量\\
\multicolumn{2}{l}{测试结果}\\
\multicolumn{2}{l}{注释}\\

\bottomrule

\end{tabulary}
\end{minipage}
\end{table}

\subsection{步骤1: 设置每页显示表格的数量}
\label{步骤1:设置每页显示表格的数量}

\begin{itemize}
\item 点击【设置】-$>$【全局设置】,设置每页显示的数量,例如设置为5,然后点击保存

\end{itemize}

\begin{figure}[htbp]
\centering
\includegraphics[keepaspectratio,width=\textwidth,height=0.75\textheight]{images/DRCT501-1.png}
\end{figure}

\subsection{步骤2: 查看列表页面中表格的显示数量是否改变}
\label{步骤2:查看列表页面中表格的显示数量是否改变}

\begin{itemize}
\item 切换到【主机保护】-$>$【已保护主机】列表页面中,查看列表页面中显示的行数是否已经改变

\end{itemize}

\begin{figure}[htbp]
\centering
\includegraphics[keepaspectratio,width=\textwidth,height=0.75\textheight]{images/DRCT501-2.png}
\end{figure}

\section{DRCT502: 语言设置}
\label{drct502:语言设置}

\begin{table}[htbp]
\begin{minipage}{\linewidth}
\setlength{\tymax}{0.5\linewidth}
\centering
\small
\begin{tabulary}{\textwidth}{@{}LL@{}} \toprule
目标&语言设置\\
\midrule
\multicolumn{2}{l}{前提条件}\\
测试步骤&1、设置DR Cloud显示语言2、查看界面显示的语言是否为英文\\
测试预期&成功设置指定语言\\
\multicolumn{2}{l}{测试结果}\\
\multicolumn{2}{l}{注释}\\

\bottomrule

\end{tabulary}
\end{minipage}
\end{table}

\subsection{步骤1: 设置每页显示表格的数量}
\label{步骤1:设置每页显示表格的数量}

\begin{itemize}
\item 点击【设置】-$>$【全局设置】,设置语言为English

\end{itemize}

\begin{figure}[htbp]
\centering
\includegraphics[keepaspectratio,width=\textwidth,height=0.75\textheight]{images/DRCT502-1.png}
\end{figure}

\subsection{步骤2: 查看界面显示的语言是否为英文}
\label{步骤2:查看界面显示的语言是否为英文}

\begin{figure}[htbp]
\centering
\includegraphics[keepaspectratio,width=\textwidth,height=0.75\textheight]{images/DRCT502-2.png}
\end{figure}
